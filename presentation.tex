\documentclass[12pt,aspectratio=169]{beamer} %for wide screen, use aspectratio=169
\usepackage{extramarks}
\usepackage{amsmath}
\usepackage{amsthm}
\usepackage{amsfonts}
\usepackage{amssymb}
\usepackage{mathrsfs}
\usepackage{tikz}
\usepackage{mathtools}
\usepackage[plain]{algorithm}
\usepackage{algpseudocode}
\usepackage{spverbatim}
\usepackage{mathabx}
\usepackage{xypic}
\usepackage{xr}

\usepackage{fancyhdr}
\usepackage{rotating}
\usepackage{booktabs}
\usepackage{graphicx}
\usepackage{bbm}
\usepackage{bm}
\usepackage{concmath}
\usepackage{cmbright}

\usepackage{listings}
\usepackage{color}
\usepackage{cancel}
\usepackage{nccmath}

\newcommand{\bbR}{\mathbb{R}}
\newcommand{\bbQ}{\mathbb{Q}}
\newcommand{\bbZ}{\mathbb{Z}}
\newcommand{\bbN}{\mathbb{N}}
\newcommand{\bbC}{\mathbb{C}}
\newcommand{\Lagr}{\mathcal{L}}
\newcommand{\ra}{\rightarrow}

\DeclareMathOperator*{\argmin}{\arg\min}
\DeclareMathOperator*{\argmax}{\arg\max}

% Useful for algorithms
\newcommand{\alg}[1]{\textsc{\bfseries \footnotesize #1}}

% For derivatives
\newcommand{\deriv}[1]{\frac{\mathrm{d}}{\mathrm{d}x} (#1)}

% For partial derivatives
\newcommand{\pderiv}[2]{\frac{\partial}{\partial #1} (#2)}

% Integral dx
\newcommand{\dx}{\mathrm{d}x}


% Probability commands: Expectation, Variance, Covariance, Bias
\newcommand{\E}{\mathbb{E}}
\newcommand{\Var}{\mathbb{V}}
\newcommand{\Prob}{\mathbb{P}}
\newcommand{\Cov}{\mathrm{Cov}}
\newcommand{\Bias}{\mathrm{Bias}}
\newcommand{\lb}{\left\lbrace}
\newcommand{\rb}{\right\rbrace}


\definecolor{dkgreen}{rgb}{0,0.6,0}
\definecolor{gray}{rgb}{0.5,0.5,0.5}
\definecolor{mauve}{rgb}{0.58,0,0.82}

\usetikzlibrary{automata,positioning}
% Specify theme
\usetheme{UnofficialUChicago}

\usefonttheme{serif}

\usepackage{beton}
\usepackage[T1]{fontenc}

 %\setbeamertemplate{footline}[frame number]{} % Uncomment this line if you want to remove the footer from each slide (and replace it with just the slide number (X/Y) in the bottom right of each slide.




%===============================================================%
% 				BEGIN YOUR PRESENTATION HERE					%
%===============================================================%

% Title and author information
\title[Contract Year Effect]{Contract Year Effect in the NBA}
\author{Charles Shi, Jonathan Liu, Terry II Culpepper}
\institute[]{University of Chicago}
\date{\today}


%  \usepackage[sfmath]{kpfonts}
%  \renewcommand*\familydefault{\sfdefault}

%\setbeamerfont{frametitle}{shape=\scshape}

%===============================================================%
\begin{document}
%===============================================================%

\maketitle



%===============================================================%
\section{Introduction}
%===============================================================%

\begin{frame}{Introduction}
Teacher unions have been a focal debate point for decades. 
\bigskip

\onslide<2,3,4>{Some argue that teacher unions act as a monopoly, decrease competition, and therefore decrease educational quality. Others argue that it brings in more money for teachers to invest in education.}

\bigskip

\onslide<3,4>{Research on this topic is ambiguous.}

\bigskip

\onslide<4>{Treat teacher unions like a collusive oligopoly - does the market power influence educational quality?}
\end{frame}

\begin{frame}{Why?}

Ambiguity on this topic suggests it may not be merely the presence of a union or lack thereof that affects educational quality. The focus on market power is our contribution to the literature.

\bigskip

\onslide<2,3>{Ex: Efficient wage hypothesis? Higher market power increases wage and teachers become more motivated.}

\bigskip

\onslide<3>{Market power =/= union power. Union power incorporates intangible effects like the internal solidarity of its members and political activism; market power takes into account the market mechanisms between teacher unions and school districts.}

\end{frame}

%===============================================================%
\section{Background}
%===============================================================%

\begin{frame}{Key Statistics}


Itemized lists are punctuated by little shields

\begin{itemize}
	\item Item
	\item Item
	\begin{itemize}
		\item Sub-item
		\item Sub-item
	\end{itemize}
	\item Item
\end{itemize}

\end{frame}

\begin{frame}{Literature}


The literature does not typically treat union behavior as firm behavior.

\begin{itemize}
	\item Focused on aspects where unions may impact teacher performance.
	\item Focused on indirect effect that unions may have on teacher performance.
	\item Literature does treat unions as a form of imperfect competition.
	\begin{itemize}
		\item Johnson and Ashenfeller (1969) - role of bargaining power
		\item Booth (2014) - imperfect labor competition arising from trade unions
	\end{itemize}
\end{itemize}

\end{frame}

\begin{frame}{Literature}


What about education?

\begin{itemize}
	\item Cowen and Strunk (2015)
	\begin{itemize}
		\item Modest negative impact on student quality
		\item Propogation of rent seeking behavior
	\end{itemize}
	\item Lott and Kenny (2013)
	\begin{itemize}
		\item Decline in student performance after negotiation with teacher unions.
	\end{itemize}
	
\end{itemize}

\end{frame}

\begin{frame}{Literature}

\begin{itemize}
	
	\item Baron (2018)
	\begin{itemize}
		\item Wisconsin Act 10 limited the power of teachers' unions
		\item In the short run, Act 10, the law reduced average test scores.
	\end{itemize}
	\item Baron (2019)
	\begin{itemize}
		\item One year on, the increase in teacher supply due to Act 10 increased average test scores.
		\item Time is a factor.
	\end{itemize}
	\item More literature showing various results
\end{itemize}
\end{frame}

%===============================================================%
\section{Market Structure}
%===============================================================%

\begin{frame}{Market Structure}
	\begin{itemize}
		\item Upstream firm: Teacher Union
		\begin{itemize}
			\item Consists of multiple teachers, acting like small, individual firms
			\item In a collusion
			\item Members may have incentive to leave the union/collusion
			\item $R$ can be thought of as the aggregate wage income
			\item $c\left(\cdot\right)$, the cost, can be thought of as
			\begin{enumerate}
				\item Disutility of labor
				\item Cost of maintaining the teacher's body and soul.
			\end{enumerate}
		\end{itemize}
	\end{itemize}
\end{frame}

\begin{frame}{Market Structure}
\begin{itemize}
	\item Downstream firm: School District
	\begin{itemize}
		\item Purchases educational services (from teachers)
		\item Maximizes profit / educational outcome on three goods:
		\begin{enumerate}
			\item Educational services
			\item Support services (schoolbus, administration)
			\item Other (food etc.)
		\end{enumerate}
	\end{itemize}
	\item Consumer: Parents \& Students
	\begin{itemize}
		\item We focus mostly on the interaction between the upstream and downstream firms: teacher union and school district.
	\end{itemize}
\end{itemize}
\end{frame}

%===============================================================%
\section{Model}
%===============================================================%

\begin{frame}{Generalized Lerner Index}
Assume:

\begin{itemize}
	\item Teachers are homogeneous in skill and quality
	\item Unionized and non-unionized teachers have similar working hours (homogeneity)
\end{itemize}

This ensures that we can use union participation rates as a proxy for market share. Then we can estimate the market power of the teacher union with the Generalized Lerner Index:
	\[
	L = -\frac{s_i}{\epsilon_d}
	\]

$L$ and $s_i$ are the market power and market share of the teacher union; $\epsilon_d$ is the elasticity of demand for educational services with respect to price.
\end{frame}

\begin{frame}{Estimating the elasticity of demand}
We use the \textbf{Almost Ideal Demand System (AIDS)} method (Deaton and Muellbauer (1980)):

School districts act as a representative firm/consumer and has expenditure function
	
	\[
	c\left(\mathbf{p}, u\right) = \left(a \left(\mathbf{p}\right)^\alpha \left(1 - u\right) + b\left(\mathbf{p}\right)^\alpha u\right)^{1/\alpha}
	\]

With some derivation we obtain a tractable demand function:

\[
w_E = \alpha_E + \sum_k \gamma_{E, k} \ln p_k + \beta_E \ln \left(\frac{w}{P}\right)
\]

where $w$ is the total expenditure, $w_E$ is the expenditure share on educational services, and $P$ is the price index.
\end{frame}

\begin{frame}{Price Index}

The price index has the formula

\[
\ln P := \alpha_0 + \sum_k \alpha_k \ln p_k + \frac{1}{2} \sum_k\sum_j \gamma_{k,j}^* \ln p_k \ln p_j.
\]

To estimate the price index, we follow Feenstra et al.'s (1999) strategy and approximate $P$ with the Divisia index:

\[
\ln P = \left[\frac{1}{6}\mathbf{w}_0 + \frac{2}{3}\mathbf{w}_{0.5} + \frac{1}{6}\mathbf{w}_1\right]\cdot \ln \left(\frac{p_1}{p_0}\right)
\]
where $p_0, p_1$ are the price index vectors for our three goods at some initial period and final period and $\mathbf{w}_i$ are the expenditure share vectors in the initial, middle, and final period. We will take the initial, middle, and final periods as consecutive years.
\end{frame}

\begin{frame}{Price Index}

For the price index for each of the three goods, we use the revenue-share-weighted average of the price of each component within this group. Ex:

\[
p_E = \sum_{i\in E} \omega_i p_i.
\]

where $\omega_i$ is the revenue share of component $i$ of good $E$ (instructional services). After the regression model which gives us $\gamma_{E, k}$, we can estimate the Marshallian self-price elasticity of educational services with

\[
\epsilon^M_E = -1 + \left(\frac{\gamma_{E,E}}{w_E}\right) - \beta_E
\]

From this, we can then estimate the market power of the teacher union with some assumptions.
\end{frame}

%===============================================================%
\section{Data Analysis}
%===============================================================%

%===============================================================%
\section{Conclusion}
%===============================================================%


{\BackgroundShaded
\begin{frame}
Questions?
\end{frame}
}



%===============================================================%
\section{References}
%===============================================================%

\begin{frame}{References}

	Bruner et al., \emph{School Finance Reforms, Teachers' Unions, and the Allocation of School Resources}. March 04 2019. https://www.mitpressjournals.org/doi/abs/10.1162/rest\_a\_00828
	\\
	
	Baron et al., \emph{The Effect of Teachers’ Unions on Student Achievement in the Short Run: Evidence from Wisconsin’s Act 10}. December 2018. https://www.sciencedirect.com/science/article/pii/S027277571830551X
	\\
	
	Baron et al., \emph{Union Reform, Performance Pay, and New Teacher Supply: Evidence from Wisconsin's Act 10}. April 19, 2019. https://papers.ssrn.com/sol3/papers.cfm?abstract\_id=3317540
	\\
	
	Cowen et al., \emph{The impact of teachers’ unions on educational outcomes: What we know and what we need to learn}. October 2015.
	\\
	
	Lott et al., \emph{State teacher union strength and student achievement}. August 2013.

\end{frame}



%===============================================================%
\end{document}
%===============================================================% 