\documentclass[12pt,aspectratio=916]{beamer} %for wide screen, use aspectratio=169
\usepackage{extramarks}
\usepackage{amsmath}
\usepackage{amsthm}
\usepackage{amsfonts}
\usepackage{amssymb}
\usepackage{mathrsfs}
\usepackage{tikz}
\usepackage{mathtools}
\usepackage[plain]{algorithm}
\usepackage{algpseudocode}
\usepackage{spverbatim}
\usepackage{mathabx}
\usepackage{xypic}
\usepackage{xr}

\usepackage{fancyhdr}
\usepackage{rotating}
\usepackage{booktabs}
\usepackage{graphicx}
\usepackage{bbm}
\usepackage{bm}
\usepackage{concmath}
\usepackage{cmbright}

\usepackage{listings}
\usepackage{color}
\usepackage{cancel}
\usepackage{nccmath}

\newcommand{\bbR}{\mathbb{R}}
\newcommand{\bbQ}{\mathbb{Q}}
\newcommand{\bbZ}{\mathbb{Z}}
\newcommand{\bbN}{\mathbb{N}}
\newcommand{\bbC}{\mathbb{C}}
\newcommand{\Lagr}{\mathcal{L}}
\newcommand{\ra}{\rightarrow}

\DeclareMathOperator*{\argmin}{\arg\min}
\DeclareMathOperator*{\argmax}{\arg\max}

% Useful for algorithms
\newcommand{\alg}[1]{\textsc{\bfseries \footnotesize #1}}

% For derivatives
\newcommand{\deriv}[1]{\frac{\mathrm{d}}{\mathrm{d}x} (#1)}

% For partial derivatives
\newcommand{\pderiv}[2]{\frac{\partial}{\partial #1} (#2)}

% Integral dx
\newcommand{\dx}{\mathrm{d}x}


% Probability commands: Expectation, Variance, Covariance, Bias
\newcommand{\E}{\mathbb{E}}
\newcommand{\Var}{\mathbb{V}}
\newcommand{\Prob}{\mathbb{P}}
\newcommand{\Cov}{\mathrm{Cov}}
\newcommand{\Bias}{\mathrm{Bias}}
\newcommand{\lb}{\left\lbrace}
\newcommand{\rb}{\right\rbrace}


\definecolor{dkgreen}{rgb}{0,0.6,0}
\definecolor{gray}{rgb}{0.5,0.5,0.5}
\definecolor{mauve}{rgb}{0.58,0,0.82}

\usetikzlibrary{automata,positioning}
% Specify theme
\usetheme{UnofficialUChicago}

\usefonttheme{serif}

\usepackage{beton}
\usepackage[T1]{fontenc}

%\setbeamertemplate{footline}[frame number]{} % Uncomment this line if you want to remove the footer from each slide (and replace it with just the slide number (X/Y) in the bottom right of each slide.




%===============================================================%
% 				BEGIN YOUR PRESENTATION HERE					%
%===============================================================%

% Title and author information
\title[Contract Year Effect]{Contract Year Effect in the NBA}
\author{Charles Shi, Jonathan Liu, Terry II Culpepper, Sean Choi}
\institute[]{University of Chicago}
\date{\today}


%  \usepackage[sfmath]{kpfonts}
%  \renewcommand*\familydefault{\sfdefault}

%\setbeamerfont{frametitle}{shape=\scshape}

%===============================================================%
\begin{document}
	%===============================================================%
	
	\maketitle
	
	
	
	%===============================================================%
	\section{Introduction}
	%===============================================================%
	
	\begin{frame}{Introduction}
	\begin{itemize}
		\item Contract Effects
		\begin{itemize}
			\item Work Ethic
			\item Overall Goal
		\end{itemize}
		\item Contract Year Phenomenon
		\begin{itemize}
			\item Players underperforming
			\item Forced effort
		\end{itemize}
		\item Performance in that year
		\begin{itemize}
			\item Sign a long term contract
		\end{itemize}
	\end{itemize}
\end{frame}


%===============================================================%
\section{Literature Review}
%===============================================================%

\begin{frame}{Literature Review}
\begin{itemize}
	\item Holmstrom (1982) and Gibbons \& Murphy (1992) discuss career concerns
	\begin{itemize}
		\item Applicable to superstars, not applicable to worse players who may mostly care about money
	\end{itemize}	
	\item Contract year phenomenon is a principal-agent problem
	
	\begin{itemize}
		\item Incentive incompatible contracts create shirking behavior and moral hazard
		\begin{itemize}
			\item Alchian and Demsetz (1972), Holmstrom (1979), Ress (1994)
		\end{itemize}			
	\end{itemize}
	\item Sports-specific research yields conflicting results of opportunistic behavior
	\begin{itemize}
		\item Lehn (182), Scogins (1993), Stiroh (2007), Paulsen (2018) find evidence of opportunistic behavior
		\item Krautmann (1990), Maxcy et al. (2002, Berri \& Krautmann (2006) do not
	\end{itemize}
	\item Conflicting results come from different tests, per Krautmann \& Donley (2009)
\end{itemize}	
\end{frame}


%===============================================================%
\section{Data Collection}
%===============================================================%

\begin{frame}{Data Collection - Salary and Advanced Stats}
\begin{itemize}
\item Scraped Salary Data and advanced stats from basketballreference.com for all current NBA players
\begin{itemize}
	\item Scraper visits the basketballreference page for every current NBA player and scrapes relevant statistics
\end{itemize}
\item Focus on the 2016-2017 to 2019-2020 regular seasons
\begin{figure}
	\centering
	\includegraphics[scale=0.3]{datacoll1.png}
\end{figure}	
\end{itemize}
\end{frame}

\begin{frame}{Data Collection - Speed and Box-Out Data}
\begin{itemize}
\item Scraped Box out and Hustle Data from stats.nba.com for all current NBA players
\begin{figure}
\centering
\includegraphics[scale=0.3]{datacoll2.png}
\end{figure}
\end{itemize}
\end{frame}

\begin{frame}{Data Collection - Speed and Box-Out Data}
\begin{figure}
\centering
\includegraphics[scale=0.3]{datacoll3.png}
\end{figure}

\end{frame}



%===============================================================%
\section{Model}
%===============================================================%

\begin{frame}{Model and Analysis}

Challenges in estimating the contract year effect fall under three categories.

\begin{itemize}
\item Player effort cannot be observed, and existing metrics such as distance covered on the field per minute may only be loosely correlated with player effort.
\item The effort put in by a particular player may also correlate with other players' performance. Since basketball is a team game, effort put in by one player may synergize with effort put in by other players.
\item Heterogeneity in different individuals: unobserved variables such as skill.
\end{itemize}
\end{frame}

\begin{frame}{Controlling for individual fixed effects}

We use the regression model

\[
y_{it} = \alpha_0 + \sum_{j=1}^N \alpha_j \mathbbm{1}\left\lbrace i= j \right\rbrace + \mathbbm{1}\left\lbrace i_t\in \mathrm{Contract} \right\rbrace\beta + c_{it}'\gamma + u_{it}
\]

Identifying assumption: unobservable effects soaked up by the individual indicator variable that simultaneously affect the outcome player statistics and the explanatory variable and covariates are time-invariant.
\end{frame}

\begin{frame}{Controlling for team fixed effects}

We use the same regression model

\[
y_{it} = \alpha_0 + \sum_{j=1}^N \alpha_j \mathbbm{1}\left\lbrace i= j \right\rbrace + x_{it}\beta + c_{it}'\gamma + u_{it}
\]

This time $x_{it}$ is the weighted number of players on a contract year.
\\

The dataset is grouped by team and the average weighted values of each variable is computed.

\end{frame}

\begin{frame}{Double LASSO: individual \& team fixed effects}

When combining datasets from three different sources, the amount of observations decrease drastically and creates a curse of dimensionality problem. To circumvent this issue we employ double LASSO.

\end{frame}

\begin{frame}{Double LASSO: individual \& team fixed effects}

\begin{enumerate}
	\item We first fit a standard Lasso regression: \[
	y_{it} = \alpha_0 + \sum_{j=1}^N \alpha_j \mathbbm{1}\left\lbrace i= j \right\rbrace + \mathbbm{1}\left\lbrace i_t\in \mathrm{Contract} \right\rbrace\beta + c_{it}'\gamma + u_{it}
	\]
	\item Drop the zero coefficient variables, then do another Lasso regression: \[
	\mathbbm{1}\left\lbrace i_t\in \mathrm{Contract} \right\rbrace = \alpha_0^L + \sum_{j=1}^N \alpha^L_j \mathbbm{1}\left\lbrace i= j \right\rbrace + c_{it}'\gamma^L + u_{it}
	\]
	\item Do a regular OLS: \[
	y_{it} = \alpha_0^O + \sum_{j=1}^N + \mathbbm{1}\left\lbrace i_t\in \mathrm{Contract} \right\rbrace\beta^O + z_{it}'\gamma^O + u_{it}
	\]
	
	where $z_{it}$ consists of covariates that are selected in step 1 and 2.
\end{enumerate}

\end{frame}

\begin{frame}{Double LASSO: individual \& team fixed effects}
	The strategy is similar for team fixed effects.

\end{frame}



%===============================================================%
\section{Data}
%===============================================================%
\small

\begin{frame}{Individual Fixed Effect OLS}

% Table created by stargazer v.5.2.2 by Marek Hlavac, Harvard University. E-mail: hlavac at fas.harvard.edu
% Date and time: Tue, Jun 02, 2020 - 10:36:20 PM
\begin{table}\centering 
	\caption{Using Average Seconds per Dribble as the Dependent Variable} 
	\label{} 
	\scalebox{0.6}{
		\begin{tabular}{@{\extracolsep{5pt}}lc} 
			\\[-1.8ex]\hline 
			\hline \\[-1.8ex] 
			& \multicolumn{1}{c}{\textit{Dependent variable:}} \\ 
			\cline{2-2} 
			\\[-1.8ex] & Average Seconds per Dribble \\ 
			\hline \\[-1.8ex] 
			Contract Year & -0.0247 \\ 
			& (0.0368) \\ 
			& \\ 
			Average Minutes Played & 0.011$^{**}$ \\ 
			& (0.00369) \\ 
			& \\ 
			Current Salary & 0.000$^{*}$ \\ 
			& (0.000) \\ 
			& \\
			Constant & 5.117$^{***}$ \\ 
			& (0.664) \\ 
			& \\ 
			\hline \\[-1.8ex] 
			Player Fixed Effects & Yes \\ 
			Year Fixed Effects & Yes \\ 
			Observations & 1,057 \\ 
			R$^{2}$ & 0.974 \\ 
			Adjusted R$^{2}$ & 0.952 \\ 
			Residual Std. Error & 1.609 (df = 569) \\ 
			F Statistic & 44.244$^{***}$ (df = 487; 569) \\ 
			\hline 
			\hline \\[-1.8ex] 
			\textit{Note:}  & \multicolumn{1}{r}{$^{*}$p$<$0.1; $^{**}$p$<$0.05; $^{***}$p$<$0.01} \\ 
	\end{tabular}} 
\end{table} 

\end{frame}

%===============================================================%
\section{Conclusion}
%===============================================================%


{\BackgroundShaded
\begin{frame}
Questions?
\end{frame}
}



%===============================================================%
\section{References}
%===============================================================%

\begin{frame}{References}


\end{frame}



%===============================================================%
\end{document}
%===============================================================% 